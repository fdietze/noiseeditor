\chapter{Vorhersage von Kompositionen}
%%\begin{figure}[ht!]
%%\rahmengraphik[0.5]{01/bla.eps}
%%\caption{ Beispiel }
%%\label{bsp_label}
%%\end{figure}
	- Intervall-Vorhersage durch Intervallarithmetik
		- Intervallarithmetik auf einfachen Funktionen
		- Intervallarithmetik für Improved Perlin Noise
			- Perlin-Noise ist Polynom
			- Polynome lassen sich in Bezierkurven mit regelmäßiger Anordnung der Kontrollpunkte umrechnen
			- Bezierkurven lassen sich leicht zerschneiden
			- Bezierkurven liegen in der Konvexen hülle der Kontrollpunkte
			- Das Minimum und Maximum der Kontrollpunkte bestimmt den zu erwartenden Wertebereich.
			- Implementierung
		- Vorhersage in der GameEngine
			- Bereichsanfrage von Oktanten
			- Kontrollpunkt-cache in der Noise-Prediction
			- Vorhersage-Code-Generierung mit dem Noise-Editor
